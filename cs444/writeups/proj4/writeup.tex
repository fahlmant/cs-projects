\documentclass[letterpaper,10pt,titlepage]{article}

\usepackage{graphicx}                                        
\usepackage{amssymb}                                         
\usepackage{amsmath}                                         
\usepackage{amsthm}                                          

\usepackage{alltt}                                           
\usepackage{float}
\usepackage{color}
\usepackage{url}

\usepackage{balance}
\usepackage[TABBOTCAP, tight]{subfigure}
\usepackage{enumitem}
\usepackage{pstricks, pst-node}

\usepackage{geometry}
\geometry{textheight=8.5in, textwidth=6in}

%random comment

\newcommand{\cred}[1]{{\color{red}#1}}
\newcommand{\cblue}[1]{{\color{blue}#1}}

\usepackage{hyperref}
\usepackage{geometry}

\def\name{Taylor Fahlman}

%% The following metadata will show up in the PDF properties
\hypersetup{
  colorlinks = true,
  urlcolor = black,
  pdfauthor = {\name},
  pdfkeywords = {cs444 ``operating systemsII''},
  pdftitle = {CS 444 Project 4},
  pdfsubject = {CS 444 Project 4},
  pdfpagemode = UseNone
}

\begin{document}

\section{PLAN}
    My plan is to fist look at how slob is implemented. Once I have a basic understanding, I will implement the system call since I can still get usefull information out of it initially. Once I've done that, I will change slob to use best fit. I'm not sure exactly what I'll change in the algorithm code wise, but I'll change slob so it doens't just allocate as soon as space is available.   
 
\section{DESIGN}
    Design was very easy overall. The most difficult part was the system calls. Most sources I found either targeted older kernels or the 64 bit version of the kernel we are using. After a bit of digging around the kernel source (and making sure slob.c got compiled), I found out where to define system calls for the 32 bit kernel, and how to call them from within the VM using the syscall systemcall. Once that was figured out, the best fit algorithm itself was pretty easy to impement. I found in slob.c where the page gets allocated, and instead of immedieatly allocating, I instead iterated through all pages until I found the smallest one that would still fit the need. Best fit got around 96 percent fragmnetation, while the default had roughly 20 percent. 


\section{REFLECTIONS}
    1. The main point is to understand how system calls work, and how memeory is managed\\
    2. I broke it up into two parts. First, figure out system calls, then figure out the best fit. \\
    3. The VM itself took signicantly longer to boot up, and it succeded. Best fit takes longer, so this was expected, and if the algorithm was incorrect, the vm wouldnt be able to boot up.\\
    4. I leared about memory management, and system call defintions.\\ 

\begin{tabular}{l l l}\textbf{Detail} & \textbf{Author} & \textbf{Description}\\\hline
\href{https://github.com/fahlmant/cs444/commit/9ab6b54724aa05b43bd97f44ce601107244f2720}{9ab6b54} & Taylor Fahlman & Added design section\\\hline
\href{https://github.com/fahlmant/cs444/commit/845f5318497342c9dcb224272cf2fd6411817907}{845f531} & Taylor Fahlman & Patch file for project 4\\\hline
\href{https://github.com/fahlmant/cs444/commit/74edb95a334ed2c03f204f3897abfa5b93fcd5c4}{74edb95} & Taylor Fahlman & Changed free unit to unsigned long to correct calculations\\\hline
\href{https://github.com/fahlmant/cs444/commit/317232474c9d8d4ee3d9cf9723015378672d901f}{3172324} & Taylor Fahlman & Calculated free space in slob_alloc\\\hline
\href{https://github.com/fahlmant/cs444/commit/4658b8c1cd4a1fb9ce30d2f21bce4c33abec0568}{4658b8c} & Taylor Fahlman & Change algorithm to find smallest page that can fit our needs\\\hline
\href{https://github.com/fahlmant/cs444/commit/f35599ba0122031a0f2174661e94be0e8aa1549f}{f35599b} & Taylor Fahlman & bad free system call, loops through each linked list and adds up free space\\\hline
\href{https://github.com/fahlmant/cs444/commit/fefb05a436e36ad3318878fc9d80459ce9a297e4}{fefb05a} & Taylor Fahlman & Slob used sys call, pagesize * number of pages\\\hline
\href{https://github.com/fahlmant/cs444/commit/f5afe2e87e38baebb3a404d318ce94c6e567e0c2}{f5afe2e} & Taylor Fahlman & add and decrememnt page count\\\hline
\href{https://github.com/fahlmant/cs444/commit/1c92a14db9ebe371b478c12d03e6d2efb378697d}{1c92a14} & Taylor Fahlman & Added system call numbers in the right place\\\hline
\href{https://github.com/fahlmant/cs444/commit/1490daa5fff5111ef989e2bbb8fb96ac9abaf8fd}{1490daa} & Taylor Fahlman & defined system calls in the kernel\\\hline
\href{https://github.com/fahlmant/cs444/commit/5fceac830d786b181830efcc7fb2d6e9bdf244c8}{5fceac8} & Taylor Fahlman & Shell of slob free\\\hline
\href{https://github.com/fahlmant/cs444/commit/fd3a8d9278138089028970d171e9bcc471f6f6de}{fd3a8d9} & Taylor Fahlman & Shell of syscall\\\hline
\href{https://github.com/fahlmant/cs444/commit/ba9033f1dbf9b77b68eab8080d30be0844024607}{ba9033f} & Taylor Fahlman & Added preliminary plan\\\hline
\href{https://github.com/fahlmant/cs444/commit/1627a4248c49937b98d1170fd66da4c1edff7ffa}{1627a42} & Taylor Fahlman & Skeleton of project 4 writeup\\\hline
\href{https://github.com/fahlmant/cs444/commit/acff9a5dd3f37331941e738ed451fdea6d24898c}{acff9a5} & Taylor Fahlman & ensure only one customer can add them selves at once, and make sure not to dereference a null pointer\\\hline
\href{https://github.com/fahlmant/cs444/commit/c47df83c33858ae9f3209159dbb61fd03dfff50b}{c47df83} & Taylor Fahlman & Filled main function with signal handling and initialized barber and customers\\\hline
\href{https://github.com/fahlmant/cs444/commit/fd7eb4166f01deda12959c4d0c40fae1ae3b2ec3}{fd7eb41} & Taylor Fahlman & Acquire and release mutex\\\hline
\href{https://github.com/fahlmant/cs444/commit/dafa10b7238c79ecd900e8788a383271aa7af3bf}{dafa10b} & Taylor Fahlman & customer calls line push\\\hline
\href{https://github.com/fahlmant/cs444/commit/9832d96404a36e464718f1ede7c89eb6720c8d41}{9832d96} & Taylor Fahlman & Moved customer incrmemnt to the push function\\\hline
\href{https://github.com/fahlmant/cs444/commit/7e8aedcdf47eeb14060dd956bd51559cb1c3bd2c}{7e8aedc} & Taylor Fahlman & Line push function, cleaned up structs so they are usable\\\hline
\href{https://github.com/fahlmant/cs444/commit/f36590940927dc233ec5ceec23c002e4a8f09e20}{f365909} & Taylor Fahlman & customer checks if there is room in the queue. If so, increments number, and calls get_hair_cut\\\hline
\href{https://github.com/fahlmant/cs444/commit/f696a4130452abc0fbdd3dfc80abcbd2396ec7e8}{f696a41} & Taylor Fahlman & Make sure the for loop actually uses the new variable\\\hline
\href{https://github.com/fahlmant/cs444/commit/7dff854b73aa058e96e2c074ca081c0883eeed2d}{7dff854} & Taylor Fahlman & Ensure that the number of customers changing doesnt affect the loop of the barber\\\hline
\href{https://github.com/fahlmant/cs444/commit/de907234be86116ce95b1674d6a89515a34bc2ff}{de90723} & Taylor Fahlman & Added fluff to cuthair and gethaircut\\\hline
\href{https://github.com/fahlmant/cs444/commit/90f2d6ab8135d2015943811d4ef02614540cf5b5}{90f2d6a} & Taylor Fahlman & Barber needs a mutex\\\hline
\href{https://github.com/fahlmant/cs444/commit/ab2a2a72b96398395dec1b84bc3b7a055fa4a1de}{ab2a2a7} & Taylor Fahlman & Changed variable name of line member to make it clear that it's a queue\\\hline
\href{https://github.com/fahlmant/cs444/commit/a4c08be2c2e95a6f642994c7dfceee5a3e15a24d}{a4c08be} & Taylor Fahlman & Implemented basic barber\\\hline
\href{https://github.com/fahlmant/cs444/commit/41904d4f66fa1a22a5d3c9ad3738074e08ecd403}{41904d4} & Taylor Fahlman & Added functions for pop, push and line and chair structs\\\hline
\href{https://github.com/fahlmant/cs444/commit/7d4489edc277768a6d69d3299970a30042c09132}{7d4489e} & Taylor Fahlman & Skeleton for cut hair\\\hline
\href{https://github.com/fahlmant/cs444/commit/3ccc68c59a6ae971199edd7637e54670b4e7f57d}{3ccc68c} & Taylor Fahlman & skelectons for barber, customer and get hair cut\\\hline
\href{https://github.com/fahlmant/cs444/commit/bdc399b4012b3133c3e21b19f8f6887c56361cd6}{bdc399b} & Taylor Fahlman & Made sure to destroy sempahore on singal catch\\\hline
\href{https://github.com/fahlmant/cs444/commit/ce630649ad60e39df8728526d8767082481121fc}{ce63064} & Taylor Fahlman & I guess the makefile changes to lpthread didnt take\\\hline
\href{https://github.com/fahlmant/cs444/commit/1807031905f4bea5c0570d63797c0d71708ac242}{1807031} & Taylor Fahlman & Filled in the process function.\\\hline
\href{https://github.com/fahlmant/cs444/commit/aa8bc1d7ab7203381cdd91be37573d3bdcccc3fb}{aa8bc1d} & Taylor Fahlman & Global counter\\\hline
\href{https://github.com/fahlmant/cs444/commit/e94a3616f29d2ae5b604fae19ddcafa4cf4d8c2b}{e94a361} & Taylor Fahlman & Make sure semaphore.h is included AND make has the correct lpthread flag\\\hline
\href{https://github.com/fahlmant/cs444/commit/c99d9e6fbc278647679bb6346721fd6651944a52}{c99d9e6} & Taylor Fahlman & Process void function init\\\hline
\href{https://github.com/fahlmant/cs444/commit/21b1f27262d1ee1d8ed6c3b676141c0bf500fccd}{21b1f27} & Taylor Fahlman & Added skeleton for con 4 part 2 and updated makefile with proper targets\\\hline
\href{https://github.com/fahlmant/cs444/commit/7a7fbb01150699f83b98bfc6807397066766a396}{7a7fbb0} & Taylor Fahlman & Added skeleton of makefile and assignment 4\\\hline
\href{https://github.com/fahlmant/cs444/commit/b3aa8442d161887dfb6a0916a8308e191390ec06}{b3aa844} & Taylor Fahlman & writeup and patches\\\hline
\href{https://github.com/fahlmant/cs444/commit/004820e9120570f2fa0627732e0de21668eaed13}{004820e} & Taylor Fahlman & Now final makefile\\\hline
\href{https://github.com/fahlmant/cs444/commit/3b6ac89e287e3eaca2b3cb1f7aca8094c2c9bf5c}{3b6ac89} & Taylor Fahlman & final makefile\\\hline
\href{https://github.com/fahlmant/cs444/commit/274304e04de3ce8f0b5e485445ff16700a98b0b0}{274304e} & Taylor Fahlman & Finished up\\\hline
\href{https://github.com/fahlmant/cs444/commit/a107a206c3be4de2a7aee2b657ae7a98eb04a68a}{a107a20} & Taylor Fahlman & Used this implementation\\\hline
\href{https://github.com/fahlmant/cs444/commit/1d76d01de43b06dbe9dab7951763ed1848c9e6c8}{1d76d01} & Taylor Fahlman & Using this page http://www.chronox.de/crypto-API, I created a cipher structre.\\\hline
\href{https://github.com/fahlmant/cs444/commit/95a5cc9c095e5a588b9813ad579cc838c5f142f7}{95a5cc9} & Taylor Fahlman & Included crypto ai\\\hline
\href{https://github.com/fahlmant/cs444/commit/bdb34702a8d5df714e0d60f94ad42670a1d9e2a6}{bdb3470} & Taylor Fahlman & Fixing make errors\\\hline
\href{https://github.com/fahlmant/cs444/commit/7750ae7de8401e83a4a038af4ae1db2884a9e71e}{7750ae7} & Taylor Fahlman & changed function calls to match the most up to date example\\\hline
\href{https://github.com/fahlmant/cs444/commit/58aab9ac4d09ff701e74675e4c780d8175c93bad}{58aab9a} & Taylor Fahlman & Remove pointer to correct make errors\\\hline
\href{https://github.com/fahlmant/cs444/commit/31dc2229f379fc0465c67536249135c513a75807}{31dc222} & Taylor Fahlman & Fixed xfer_bio based on new code\\\hline
\href{https://github.com/fahlmant/cs444/commit/20dad183ac4d3f832c3897a2074652a697506bbc}{20dad18} & Taylor Fahlman & Moved sbd to bottom in case of dependencies\\\hline
\href{https://github.com/fahlmant/cs444/commit/ee7413dc03341028540620375482c5a7e72f729c}{ee7413d} & Taylor Fahlman & Found much newer sbull implementation, using that\\\hline
\href{https://github.com/fahlmant/cs444/commit/5f9ca5b506e502aaa1e4103dbdc7c91220471bd6}{5f9ca5b} & Taylor Fahlman & More cleanup from ldd3\\\hline
\href{https://github.com/fahlmant/cs444/commit/613f79c8756dc8338c7749a141e23f270ec615fd}{613f79c} & Taylor Fahlman & Added target to makefile\\\hline
\href{https://github.com/fahlmant/cs444/commit/812e57e06b0a842babf05c78b43413c10697b97e}{812e57e} & Taylor Fahlman & udpated to change deprocated functions\\\hline
\href{https://github.com/fahlmant/cs444/commit/ecc9f2a339b8b574a762b1329a18bb9cd677318e}{ecc9f2a} & Taylor Fahlman & Kconfig target\\\hline
\href{https://github.com/fahlmant/cs444/commit/2acc30cce09a4737ddc47dae0858626e3452e5d3}{2acc30c} & Taylor Fahlman & sbd target in makefile\\\hline
\href{https://github.com/fahlmant/cs444/commit/f1c33256670dd241e6017234879534c7039a3032}{f1c3325} & Taylor Fahlman & Invalidate, ldd3\\\hline
\href{https://github.com/fahlmant/cs444/commit/9b782d93b8f054a5c1cf1e61b5cf40183f3857e7}{9b782d9} & Taylor Fahlman & sbd exit, from ldd3 full implementation\\\hline
\href{https://github.com/fahlmant/cs444/commit/a0a30d2c9c9fd9f76993f86a59a6195c571db27b}{a0a30d2} & Taylor Fahlman & Completed init based off full ldd3 implementation\\\hline
\href{https://github.com/fahlmant/cs444/commit/a5bbcbfe1f20b45f2902911217caeead7b5230e9}{a5bbcbf} & Taylor Fahlman & simple request, from ldd3\\\hline
\href{https://github.com/fahlmant/cs444/commit/c3370db316fc781f54c9b53345cd596653ce68d3}{c3370db} & Taylor Fahlman & oops, moved the code to the real function\\\hline
\href{https://github.com/fahlmant/cs444/commit/ea44a818a5d009be8ea588c0457ee154adadd8fe}{ea44a81} & Taylor Fahlman & Handle requests without queue, from ldd3\\\hline
\href{https://github.com/fahlmant/cs444/commit/556b807f9df3c2511c99834f90a7db474be95ed6}{556b807} & Taylor Fahlman & xfer_bio, from ldd3\\\hline
\href{https://github.com/fahlmant/cs444/commit/043e5496aa11e70af81a109354acf313e04546f3}{043e549} & Taylor Fahlman & xfer_request, from ldd3\\\hline
\href{https://github.com/fahlmant/cs444/commit/379346f09f81f25716488cb6f078b36eb55ad95b}{379346f} & Taylor Fahlman & Full request, ldd3\\\hline
\href{https://github.com/fahlmant/cs444/commit/c72f15f69feb16d4c5c931a1ac1fa7bee9f50e54}{c72f15f} & Taylor Fahlman & transfer, controls size of copy\\\hline
\href{https://github.com/fahlmant/cs444/commit/78b54a5482e78e99c49671fc1bdfc4220cd15e69}{78b54a5} & Taylor Fahlman & Revalidate, from LDD3\\\hline
\href{https://github.com/fahlmant/cs444/commit/72ad4fb072c815d6c043cda034900290b946487f}{72ad4fb} & Taylor Fahlman & commit 69c6d65e9ca64cf710b2395fb93e8503a453b177 Author: Taylor Fahlman <you@example.com> Date:   Fri Nov 6 18:40:56 2015 -0800\\\hline
\href{https://github.com/fahlmant/cs444/commit/d55496cb5992fb77bf617407c1780b2bda22e95b}{d55496c} & Taylor Fahlman & Added skeleton sbd\\\hline
\href{https://github.com/fahlmant/cs444/commit/b24b56fd7a67cd25cae6c0711a306da74a8d7afd}{b24b56f} & Taylor Fahlman & Making sure I have a copy of sstf-iosched\\\hline
\href{https://github.com/fahlmant/cs444/commit/d9422f60653bf4faec3d8f4d7956d74635c64ec1}{d9422f6} & Taylor Fahlman & Cleaning up before proejct 3\\\hline
\href{https://github.com/fahlmant/cs444/commit/98906f41c1901e64ccfbd48175591f9cb4e3e564}{98906f4} & Taylor Fahlman & Actually increased searchers\\\hline
\href{https://github.com/fahlmant/cs444/commit/f1966525e032f176f800cde247991a191c6e96e8}{f196652} & Taylor Fahlman & Added explanitory comments\\\hline
\href{https://github.com/fahlmant/cs444/commit/d4072075c5ac343e9ba1d4fd53da5062d2db6b60}{d407207} & Taylor Fahlman & trying to stop segfault\\\hline
\href{https://github.com/fahlmant/cs444/commit/646563dfc803868c671fc00b9d63418021f4bfe2}{646563d} & Taylor Fahlman & Started threads\\\hline
\href{https://github.com/fahlmant/cs444/commit/5ed8151b742665756b9fde5e531f37ef0e50df03}{5ed8151} & Taylor Fahlman & Created base threads and function pointers\\\hline
\href{https://github.com/fahlmant/cs444/commit/4e7f0888594cdcc6ca918e340622a0c292855ef2}{4e7f088} & Taylor Fahlman & searcher too\\\hline
\href{https://github.com/fahlmant/cs444/commit/127ed1fa55392ce67c5082eed8e2f7a73db2026b}{127ed1f} & Taylor Fahlman & Need to make sure delete doesnt delete a non existant node\\\hline
\href{https://github.com/fahlmant/cs444/commit/2cf17628d8ca7888520cf9403f2577f57806bd71}{2cf1762} & Taylor Fahlman & Searcher waits until there is one deleter\\\hline
\href{https://github.com/fahlmant/cs444/commit/d85c03a6b857f72891efa170e38dc11395d77ae4}{d85c03a} & Taylor Fahlman & Inserter waits until it's the only inserter and there are no deleters\\\hline
\href{https://github.com/fahlmant/cs444/commit/c485a95c353e793a843ad6bdac4eea06562e642a}{c485a95} & Taylor Fahlman & Deleter now waits until it is the only thread running\\\hline
\href{https://github.com/fahlmant/cs444/commit/7febf0eb7674a18771178778a1b1b246e9722602}{7febf0e} & Taylor Fahlman & Created global counters for threads\\\hline
\href{https://github.com/fahlmant/cs444/commit/c9d63807d789e377e196ce1a94e3320a7514a901}{c9d6380} & Taylor Fahlman & Added deleter\\\hline
\href{https://github.com/fahlmant/cs444/commit/c851c1dc8be7c97ca7bbbe94fa9d80150763d259}{c851c1d} & Taylor Fahlman & Restructured a bit, and it works now\\\hline
\href{https://github.com/fahlmant/cs444/commit/55cb7dc0b437c2cc03c1a7683558cd48dd96c634}{55cb7dc} & Taylor Fahlman & Changed name of argument for threads\\\hline
\href{https://github.com/fahlmant/cs444/commit/c5dc7e1f2b65460a3e119935f06968ed0f8b2691}{c5dc7e1} & Taylor Fahlman & Added another struct to handle all the arguments the threads will ened\\\hline
\href{https://github.com/fahlmant/cs444/commit/cd77e11202390d5f7cdad25b2e5ecfa5d42f7dcd}{cd77e11} & Taylor Fahlman & Void number argument to deleter to count number of links\\\hline
\href{https://github.com/fahlmant/cs444/commit/6a3ca86d0ff1a6e3894d9f03838f423564d0a755}{6a3ca86} & Taylor Fahlman & Get out of here\\\hline
\href{https://github.com/fahlmant/cs444/commit/23140fd4536d1b8552c0db457dcfa4e2b8161635}{23140fd} & Taylor Fahlman & Infinite for loop in main\\\hline
\href{https://github.com/fahlmant/cs444/commit/851e6f51a0b171ec4deac5923320cae97a549be6}{851e6f5} & Taylor Fahlman & Track number of links with item member variable, corrected makefile issues\\\hline
\href{https://github.com/fahlmant/cs444/commit/1b669a7f2ee9d13ba60e92be4d83db9569866bfc}{1b669a7} & Taylor Fahlman & Added void argument to searcher so it can know how many links to traverse\\\hline
\href{https://github.com/fahlmant/cs444/commit/78be4afe0b06da4185d41362312d3d124971ac9c}{78be4af} & Taylor Fahlman & inserter skeleton\\\hline
\href{https://github.com/fahlmant/cs444/commit/f996c006d61d3b31ba4570b25c11d30054dd0bd3}{f996c00} & Taylor Fahlman & Added global buffer\\\hline
\href{https://github.com/fahlmant/cs444/commit/d78b59016da477c48d733d257f3ff1f218cad153}{d78b590} & Taylor Fahlman & created linked list and linked list item\\\hline
\href{https://github.com/fahlmant/cs444/commit/3a339f3c25f472320ff955659e62ab5bef342bd1}{3a339f3} & Taylor Fahlman & Changed make clean to target the a3 executable\\\hline
\href{https://github.com/fahlmant/cs444/commit/b204075045d96c979c5f23c948a7aa96f7593ea5}{b204075} & Taylor Fahlman & Outline of searcher, inserter and deleter functions\\\hline
\href{https://github.com/fahlmant/cs444/commit/e88db0870f79e0fda434105d2b1a43ffe33248d0}{e88db08} & Taylor Fahlman & Outline for concurrency 3\\\hline
\href{https://github.com/fahlmant/cs444/commit/6da20559bf2fd5648a71e263be1fcddb68cb1786}{6da2055} & Taylor Fahlman & Added patch file\\\hline
\href{https://github.com/fahlmant/cs444/commit/7e4a6fad9eede1051a7ebc5db14bd3f360077816}{7e4a6fa} & Taylor Fahlman & reset file to semi-working copy\\\hline
\href{https://github.com/fahlmant/cs444/commit/886806cc8700041836984270d8457cb305283982}{886806c} & Taylor & One more cleanup\\\hline
\href{https://github.com/fahlmant/cs444/commit/d1c8f5c0597b0374c8fa150baf8654161f50eb83}{d1c8f5c} & Taylor & Updated writeup, cleaned up other files\\\hline
\href{https://github.com/fahlmant/cs444/commit/337f55d1a7a77b5840ab7cd55d3b98e6f8a4ea61}{337f55d} & Taylor Fahlman & Added writeup file\\\hline
\href{https://github.com/fahlmant/cs444/commit/16dfafe2b2ddaa55c00dc6f13d1624f5f348abdc}{16dfafe} & Taylor Fahlman & Added debugging statements\\\hline
\href{https://github.com/fahlmant/cs444/commit/4e313b07a7ce52c2eeb15e77780b0344fd58e776}{4e313b0} & Taylor Fahlman & Fixed init to match closer to noop, added printk for debugging\\\hline
\href{https://github.com/fahlmant/cs444/commit/48b1e091f6f8445fc355268ded9d2ab085f38217}{48b1e09} & Taylor Fahlman & Initialized the queue going forward:\\\hline
\href{https://github.com/fahlmant/cs444/commit/ee5d1be765a4359e795c1cfab79f45e7157dc427}{ee5d1be} & Taylor Fahlman & Fixed make errors\\\hline
\href{https://github.com/fahlmant/cs444/commit/19b5eed5fce8f0988d6de8e32433e71ec722f401}{19b5eed} & Taylor Fahlman & Fixed prink\\\hline
\href{https://github.com/fahlmant/cs444/commit/1b8337adbfd556c19331c13e47fdf30f6be4aa2c}{1b8337a} & Taylor Fahlman & Add sorter to put request in correct place and add prink statement\\\hline
\href{https://github.com/fahlmant/cs444/commit/0a48e33e7e336bcfa08225cb01f3df0fcaaed1f8}{0a48e33} & Taylor Fahlman & Add sorter to put request in correct place\\\hline
\href{https://github.com/fahlmant/cs444/commit/af1472c36acd59ba67588633806bf0893cfe8b8b}{af1472c} & Taylor Fahlman & Set request sector position\\\hline
\href{https://github.com/fahlmant/cs444/commit/4e4e0d50bb79cb0287dd9e15bf858950aca4e52c}{4e4e0d5} & Taylor Fahlman & Added prev and next request definitions\\\hline
\href{https://github.com/fahlmant/cs444/commit/264706d970dd72869b019450b1f75bc089303f51}{264706d} & Taylor Fahlman & Handled add request if the list was empty\\\hline
\href{https://github.com/fahlmant/cs444/commit/96e8b1364367fb70321afbf63863220dec0966bb}{96e8b13} & Taylor Fahlman & Added variables to add request\\\hline
\href{https://github.com/fahlmant/cs444/commit/09cf9a81142f457cc824439568eb4f7d0202b485}{09cf9a8} & Taylor Fahlman & Actually process request\\\hline
\href{https://github.com/fahlmant/cs444/commit/8fa13421d3f213f3e1698235bd37426b314838d6}{8fa1342} & Taylor Fahlman & Added print statement for debugging/proff\\\hline
\href{https://github.com/fahlmant/cs444/commit/2821d6ce67d5d1a5051838079defe48c4f14ee9f}{2821d6c} & Taylor Fahlman & Fixed makefile errors\\\hline
\href{https://github.com/fahlmant/cs444/commit/e9341bad2ae0d815c1eddedcb0070bb99c3ae231}{e9341ba} & Taylor Fahlman & Implemented backwards\\\hline
\href{https://github.com/fahlmant/cs444/commit/fc70a2c72e4933dbf271bf6f5b4adc250cd93f0b}{fc70a2c} & Taylor Fahlman & Handled going forward in requests\\\hline
\href{https://github.com/fahlmant/cs444/commit/95ceb4f9438107951331d241099b020416a2e580}{95ceb4f} & Taylor Fahlman & Checked if previous and next request are the same\\\hline
\href{https://github.com/fahlmant/cs444/commit/eea31387ddd4b439198607910a6025a957d4ce2b}{eea3138} & Taylor Fahlman & Commented to add notes\\\hline
\href{https://github.com/fahlmant/cs444/commit/a2427eaa94cfc72c3c636d359e1ff18729c50803}{a2427ea} & Taylor Fahlman & If list is not empty, create next and previous request pointers\\\hline
\href{https://github.com/fahlmant/cs444/commit/179f677a63ce6e59233d66e1696cc2a2c6da4f94}{179f677} & Taylor Fahlman & Added exit queue\\\hline
\href{https://github.com/fahlmant/cs444/commit/ef0f078626f6b3779db8239cfadefc9b0eb57422}{ef0f078} & Taylor Fahlman & Hopefully fixed strange differences between local and master\\\hline
\href{https://github.com/fahlmant/cs444/commit/e465e8bca4a176e46eec39c9506fcbbeaa52346d}{e465e8b} & Taylor Fahlman & Fixed make errors\\\hline
\href{https://github.com/fahlmant/cs444/commit/90fe379dcec284aec20892a7b5dfed7418d1f629}{90fe379} & Taylor Fahlman & Corrected typo fixes and makefile errors\\\hline
\href{https://github.com/fahlmant/cs444/commit/474cbfd6cf240364ec4739b91d21f0a08efa0507}{474cbfd} & Taylor Fahlman & Added struct and shell functions\\\hline
\href{https://github.com/fahlmant/cs444/commit/9a276f27141dec3a008c5ff9b3f53302579d55dc}{9a276f2} & Taylor Fahlman & added target to makefile\\\hline
\href{https://github.com/fahlmant/cs444/commit/41cc4baab6306e573e7cebb13ab23eba9ee28942}{41cc4ba} & Taylor Fahlman & Okay, really fixed formatting\\\hline
\href{https://github.com/fahlmant/cs444/commit/f744814a5f73441d92f08e7ec4c1734b8effe233}{f744814} & Taylor Fahlman & fixed formattin?\\\hline
\href{https://github.com/fahlmant/cs444/commit/3f40ac2f576b2250e924524e3c7973a7b57b99dc}{3f40ac2} & Taylor Fahlman & added look to default selection\\\hline
\href{https://github.com/fahlmant/cs444/commit/b49de190dda284a50bb2738f40b8436171b9d691}{b49de19} & Taylor Fahlman & Added inital sstf-iosched file and module init/exit\\\hline
\href{https://github.com/fahlmant/cs444/commit/1fd55679562034d28fcfbb6e2ac1642c5c88dfc9}{1fd5567} & Taylor Fahlman & Added LOOK as an option in the I/O scheduling config file\\\hline
\href{https://github.com/fahlmant/cs444/commit/d9c6b45e2f1eee4649ad42b115e750bd68d2622a}{d9c6b45} & Taylor Fahlman & Updated make clean\\\hline
\href{https://github.com/fahlmant/cs444/commit/f93baa8e24785744b1f3ae51ec64e45e62e6ea6e}{f93baa8} & Taylor Fahlman & Removed all pthread conditions\\\hline
\href{https://github.com/fahlmant/cs444/commit/2021542aebdf6ed1edee48fec117909c7b9926da}{2021542} & Taylor Fahlman & Changed control construct to while instead of if\\\hline
\href{https://github.com/fahlmant/cs444/commit/3a97f9ddf3e62cf13f868b6554fa48b76e200e64}{3a97f9d} & Taylor Fahlman & Added pritnf statements\\\hline
\href{https://github.com/fahlmant/cs444/commit/6ac82c37f00e5ac3d70cc9fce0cdd69a4b09a1c0}{6ac82c3} & Taylor Fahlman & Added print statements for plato and locke\\\hline
\href{https://github.com/fahlmant/cs444/commit/55e26ff34cc7fc3159c10da1add5024754667dbd}{55e26ff} & Taylor Fahlman & Added marx thread\\\hline
\href{https://github.com/fahlmant/cs444/commit/88de25a62a2e699650bc207419b2b384abc5a748}{88de25a} & Taylor Fahlman & Removed swp file\\\hline
\href{https://github.com/fahlmant/cs444/commit/ec8a5fbea281d6bffd2f3df386889152fadb13bc}{ec8a5fb} & Taylor Fahlman & Added marx function\\\hline
\href{https://github.com/fahlmant/cs444/commit/ba1a2150d205c820597bf84f89c025edfb785238}{ba1a215} & Taylor Fahlman & Added socrates thread\\\hline
\href{https://github.com/fahlmant/cs444/commit/edb136fac1c4ed5c5317f7a7a967bad090b32c86}{edb136f} & Taylor Fahlman & Added socrates function\\\hline
\href{https://github.com/fahlmant/cs444/commit/3cf1db8bc4cada1b6be913bc35423bd7d5f15855}{3cf1db8} & Taylor Fahlman & Added pythag thread\\\hline
\href{https://github.com/fahlmant/cs444/commit/d9a65baf438b66b6e0a9fd4cbfdf14bbb4244d5c}{d9a65ba} & Taylor Fahlman & Added pythagorus function\\\hline
\href{https://github.com/fahlmant/cs444/commit/8a6cceb7c92a712f5428de6f30d8c77f80e8dc16}{8a6cceb} & Taylor Fahlman & Added locke thread\\\hline
\href{https://github.com/fahlmant/cs444/commit/58ca58d458e413a75a8c02487bb65cf6acef3d59}{58ca58d} & Taylor Fahlman & Filled locke function\\\hline
\href{https://github.com/fahlmant/cs444/commit/ebf0d4a0840a6a6bb8b17916dc160c469197593d}{ebf0d4a} & Taylor Fahlman & Cleaned up after signal catch\\\hline
\href{https://github.com/fahlmant/cs444/commit/b48fefd146280d909c6cf2155b3fe1fdec0a3974}{b48fefd} & Taylor Fahlman & Added mt19937.h\\\hline
\href{https://github.com/fahlmant/cs444/commit/5e17f747c9f116ab75a330d8ce0bfdb3f3ec35ab}{5e17f74} & Taylor Fahlman & Added random num generator, eat, and think functions\\\hline
\href{https://github.com/fahlmant/cs444/commit/3426649ae1918162e2ffc19e8b76ffebcbfeafb6}{3426649} & Taylor Fahlman & Added infinite for loop\\\hline
\href{https://github.com/fahlmant/cs444/commit/f76dbb79c66c9c07cf4077e12042e9a2e0fd820f}{f76dbb7} & Taylor Fahlman & Added function pointer for proper pthread creation\\\hline
\href{https://github.com/fahlmant/cs444/commit/51d58b7179ad83063834c6bf5eac684dbe2dcea1}{51d58b7} & Taylor Fahlman & Added plato thread\\\hline
\href{https://github.com/fahlmant/cs444/commit/ffa8a6dcce81ffcaa6b0953ac7e66c13a5eebf23}{ffa8a6d} & Taylor Fahlman & Added pthread condition signals\\\hline
\href{https://github.com/fahlmant/cs444/commit/0b270f9d732ab6a652aaba7f6947f13c0e9cdc46}{0b270f9} & Taylor Fahlman & Added flag checking for forks\\\hline
\href{https://github.com/fahlmant/cs444/commit/b2297cd2ec70c735e39a1cc26d5ff39478f011bf}{b2297cd} & Taylor Fahlman & Restructed to use infinite loop\\\hline
\href{https://github.com/fahlmant/cs444/commit/0773ca18437eb1aff2874e73ff18fe67681602dc}{0773ca1} & Taylor Fahlman & Changed mutex pairs to int flags, and initiliazed mutexs\\\hline
\href{https://github.com/fahlmant/cs444/commit/1f5b835fc8b41d44ee33a2f2e6f5ec916591957a}{1f5b835} & Taylor Fahlman & Added mutex locks/unlocks to plato\\\hline
\href{https://github.com/fahlmant/cs444/commit/5d024e0d8a49d0db3b239929402e049376df57e1}{5d024e0} & Taylor Fahlman & added eat and think functions\\\hline
\href{https://github.com/fahlmant/cs444/commit/4795134ae4187793f395617ba52841d7c9e7f6d1}{4795134} & Taylor Fahlman & Labled philosophers\\\hline
\href{https://github.com/fahlmant/cs444/commit/d43a10f1ac0ff2b53122ca42969b95d5e5b474b9}{d43a10f} & Taylor Fahlman & added mutexes\\\hline
\href{https://github.com/fahlmant/cs444/commit/13edff973f75105ee082af2453bb0ea1789c6823}{13edff9} & Taylor Fahlman & Fixed merge conflicts\\\hline
\href{https://github.com/fahlmant/cs444/commit/084ab1563492ad141bbbf835df613426e6070df2}{084ab15} & Taylor Fahlman & Moved assignment 1 to subdir, added assignment2\\\hline
\href{https://github.com/fahlmant/cs444/commit/e9a06375c5e5ee5232d5a3a71fa1dd1ac336b7cc}{e9a0637} & Taylor & Finished writeup tex and pdf\\\hline
\href{https://github.com/fahlmant/cs444/commit/27eb37349447092eec979e1d09531408f38ac661}{27eb373} & Taylor Fahlman & Added bash file to transform gitlog to latex. Added in makefile targets\\\hline
\href{https://github.com/fahlmant/cs444/commit/2d454270c6b79091f4c1363da7185e3ffbcb2301}{2d45427} & Taylor Fahlman & Added writeup skeleton\\\hline
\href{https://github.com/fahlmant/cs444/commit/629da73d1a287be9db4919d9bc27c987acfb7128}{629da73} & Taylor Fahlman & Merge branch 'master' of https://github.com/fahlmant/cs444\\\hline
\href{https://github.com/fahlmant/cs444/commit/c96174c136e9194fd83f4d5fee5927432468fdfd}{c96174c} & Taylor Fahlman & Fixed tabbing\\\hline
\href{https://github.com/fahlmant/cs444/commit/6a7edbe421fb5e53a9a847a09117500c1ca1482e}{6a7edbe} & Taylor & Fixed blocking on full buffer\\\hline
\href{https://github.com/fahlmant/cs444/commit/007816390983ac7d562e8f9d4d31ff6abc7c40c8}{0078163} & Taylor & Removed producer sleep time because I couldn't get it to work\\\hline
\href{https://github.com/fahlmant/cs444/commit/9fb3e850a20f58cd174a80bf580c150ebb8a8860}{9fb3e85} & Taylor Fahlman & Added mersene twister files\\\hline
\href{https://github.com/fahlmant/cs444/commit/8e614a90d9b3adfedea1214253801c36f17c6aea}{8e614a9} & Taylor Fahlman & Added mersene twister generation. Added more cleanup in the signal catch. Moved pthread create to infinite loops so that it would get more numbers from the buffer\\\hline
\href{https://github.com/fahlmant/cs444/commit/6bc6226e01a9a1302c741331cce8ab8256f399ce}{6bc6226} & Taylor Fahlman & Added blocking on full or empty buffer. Added calls to random number generation.\\\hline
\href{https://github.com/fahlmant/cs444/commit/4bcd392bf2941eb49cc82229be9495cd4c9f62ee}{4bcd392} & Taylor Fahlman & Added pthread condition variables, and fixed rdrand macro\\\hline
\href{https://github.com/fahlmant/cs444/commit/f927d9b39907620e959c235c9b56ab0160bbc366}{f927d9b} & Taylor Fahlman & Fixed prototype of number generator function\\\hline
\href{https://github.com/fahlmant/cs444/commit/056005ccb725e1ce91c211bb5cc71acdd3cfe5e4}{056005c} & Taylor Fahlman & Fixed cpuid function\\\hline
\href{https://github.com/fahlmant/cs444/commit/97b39ebde5ea4090c9371d7868ec9984c6230703}{97b39eb} & Taylor Fahlman & Corrected CPUID define to differ from other global variables. Added rdrand generate define\\\hline
\href{https://github.com/fahlmant/cs444/commit/12d413cdbc773bea9797fbc8ffc523df368cae5c}{12d413c} & Taylor Fahlman & Added CPUID definition and fixed compiler errors and warning about function definitions\\\hline
\href{https://github.com/fahlmant/cs444/commit/a61e6320173be1ea668966396bb45bc7930d34f9}{a61e632} & Taylor Fahlman & Removed return that will never be executed\\\hline
\href{https://github.com/fahlmant/cs444/commit/4d08be28f8900f197846be9ac81d7c0d4c45cc03}{4d08be2} & Taylor Fahlman & Moved function defintions to correct place\\\hline
\href{https://github.com/fahlmant/cs444/commit/1a04be2d025bdf2f264d2fdcd2dad95ded4fb324}{1a04be2} & Taylor & Added random number generation function\\\hline
\href{https://github.com/fahlmant/cs444/commit/95628a0549663a27c0a3ef1f04118db28e89acda}{95628a0} & Taylor & Added signal.h so signals can work\\\hline
\href{https://github.com/fahlmant/cs444/commit/a4f7c992ec397d9b510126246137537b77e55930}{a4f7c99} & Taylor & typo fixes\\\hline
\href{https://github.com/fahlmant/cs444/commit/29e0fd1c29a38fec3bd5a999d8bf8efa513c70fd}{29e0fd1} & Taylor & Added signal catch and reimplemented buffer as queue\\\hline
\href{https://github.com/fahlmant/cs444/commit/2274e684608cf836dc48bda6d3ec5d5d9b66f5fb}{2274e68} & Taylor Fahlman & Added mutex lock/unlock to consumer\\\hline
\href{https://github.com/fahlmant/cs444/commit/7249be01ba7edbda6723b5acbe59f46efa2886a9}{7249be0} & Taylor Fahlman & Fixed issues with printing to stdout\\\hline
\href{https://github.com/fahlmant/cs444/commit/050904eeb0342d93e297fec347a08544cc99ce07}{050904e} & Taylor Fahlman & Added mutex init, and locking/releasing on produce\\\hline
\href{https://github.com/fahlmant/cs444/commit/50cf97754d7e3aa081ef30f56af307d2671969ff}{50cf977} & Taylor Fahlman & Changed global buffer pointer to a variable. Changed reference to it so that it could be written to and read from\\\hline
\href{https://github.com/fahlmant/cs444/commit/c5c13a973ba6f8363baf95b9e2330ea4dca16651}{c5c13a9} & Taylor Fahlman & Changed pthread create functions to not segfault\\\hline
\href{https://github.com/fahlmant/cs444/commit/7ff17bdada1706eb1057ba678bde47c8f5b18774}{7ff17bd} & Taylor Fahlman & Outlined needed steps in produce\\\hline
\href{https://github.com/fahlmant/cs444/commit/d09e23095844c4b70beb8f084938a6716eb11f24}{d09e230} & Taylor Fahlman & Changed the global buffer to a struct\\\hline
\href{https://github.com/fahlmant/cs444/commit/209f7513f40020a3ae61565dd7f115b7c72b8472}{209f751} & Taylor Fahlman & Added comments for placeholders and fixed print statement\\\hline
\href{https://github.com/fahlmant/cs444/commit/5d9edbd4229b7263d39225f1ec581258af623a3d}{5d9edbd} & Taylor Fahlman & Removed unneeded semaphore .h>\\\hline
\href{https://github.com/fahlmant/cs444/commit/2c5734f5d2cbbf0650eefb5acbf3da30c458f406}{2c5734f} & Taylor Fahlman & Updated assignment 1\\\hline
\href{https://github.com/fahlmant/cs444/commit/4b56781b7d05dd3c5653e92d8b3bfe43374a4190}{4b56781} & Taylor Fahlman & Changed the buffer to be a pointer to an array of 32 buffer items\\\hline
\href{https://github.com/fahlmant/cs444/commit/6445be50d9acb170ec3adaa80fe9418b160faf23}{6445be5} & Taylor Fahlman & Added sleep and print in consume function\\\hline
\href{https://github.com/fahlmant/cs444/commit/f4facc9d5a16eb42168ac97078b0f487a08a407c}{f4facc9} & Taylor Fahlman & Fixed compile issues\\\hline
\href{https://github.com/fahlmant/cs444/commit/607fa54aeb02833e23feaaebe3ca42504e86a20c}{607fa54} & Taylor Fahlman & Filled consume function\\\hline
\href{https://github.com/fahlmant/cs444/commit/2244c720455b015276a94f97610bd3e710e41705}{2244c72} & Taylor Fahlman & Assigned buffer pointer correctly to the address of the item with dummy values\\\hline
\href{https://github.com/fahlmant/cs444/commit/4d9e3d84c0309caf5af3ca5aed2af08010b2ef85}{4d9e3d8} & Taylor Fahlman & Filled general outline of produce function\\\hline
\href{https://github.com/fahlmant/cs444/commit/e2d0eedfaec7397cf59d7f1fd46886026f0f1727}{e2d0eed} & Taylor Fahlman & Makefile and assignment1.c update\\\hline
\href{https://github.com/fahlmant/cs444/commit/d4c0a889450f5e33eeee262f56cea203a29aed0a}{d4c0a88} & Taylor Fahlman & Makefile:     Added make file will the most common flags for style and debugging. Used a two step compile process, .o then an executable\\\hline
\href{https://github.com/fahlmant/cs444/commit/15a4c12977816ab8bab1fe5f5a9b3c3774a8367a}{15a4c12} & Your Name & Added concurrency\\\hline
\href{https://github.com/fahlmant/cs444/commit/1ca41740bd9248799f23d5798a4a82d99e63ba5a}{1ca4174} & Your Name & Added linux kernel\\\hline\end{tabular}

\end{document}
