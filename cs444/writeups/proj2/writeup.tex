\documentclass[letterpaper,10pt,titlepage]{article}

\usepackage{graphicx}                                        
\usepackage{amssymb}                                         
\usepackage{amsmath}                                         
\usepackage{amsthm}                                          

\usepackage{alltt}                                           
\usepackage{float}
\usepackage{color}
\usepackage{url}

\usepackage{balance}
\usepackage[TABBOTCAP, tight]{subfigure}
\usepackage{enumitem}
\usepackage{pstricks, pst-node}

\usepackage{geometry}
\geometry{textheight=8.5in, textwidth=6in}

%random comment

\newcommand{\cred}[1]{{\color{red}#1}}
\newcommand{\cblue}[1]{{\color{blue}#1}}

\usepackage{hyperref}
\usepackage{geometry}

\def\name{Taylor Fahlman}

%% The following metadata will show up in the PDF properties
\hypersetup{
  colorlinks = true,
  urlcolor = black,
  pdfauthor = {\name},
  pdfkeywords = {cs444 ``operating systemsII''},
  pdftitle = {CS 444 Project 1},
  pdfsubject = {CS 444 Project 1},
  pdfpagemode = UseNone
}

\begin{document}

\section{PLAN}
    My plan was to first look at no-op to understand IO schedulers work. I looked at the modularization of the functions, and saw
    that dispatch executed the request while add managed putting the request in the correct place. I then drew out how the add and
    dispatch function should work on paper. Then using no-op af a reference, I used the correct struct, etc. to finish the
    implementation.

\section{REFLECTIONS}
    1. I think the point of the assignment was to let us figure out how to use kernel modules, and figure out complex code ourselves.\\
    2. I approached the problem in two parts. First, how did elevator work. Once I figured that out, I moved on to figuring out how to implement it and include it in the kernel.\\
    3. I made sure it compiled, make sure the kernel used it as default by setting options, fixed many kernel panics, and used printk statements to show my code was actually running. Doing basic file operations worked. \\
    4. I learned how Kernel moduels work, and a lot about Disk I/O and kernel data structures. \\ 

\begin{tabular}{l l l}\textbf{Detail} & \textbf{Author} & \textbf{Description}\\\hline
\href{https://github.com/fahlmant/cs444/commit/1ca41740bd9248799f23d5798a4a82d99e63ba5a}{1ca4174} & Your Name & Added linux kernel\\\hline
\href{https://github.com/fahlmant/cs444/commit/15a4c12977816ab8bab1fe5f5a9b3c3774a8367a}{15a4c12} & Your Name & Added concurrency\\\hline
\href{https://github.com/fahlmant/cs444/commit/d4c0a889450f5e33eeee262f56cea203a29aed0a}{d4c0a88} & Taylor Fahlman & Makefile:     Added make file will the most common flags for style and debugging. Used a two step compile process, .o then an executable\\\hline
\href{https://github.com/fahlmant/cs444/commit/e2d0eedfaec7397cf59d7f1fd46886026f0f1727}{e2d0eed} & Taylor Fahlman & Makefile and assignment1.c update\\\hline
\href{https://github.com/fahlmant/cs444/commit/4d9e3d84c0309caf5af3ca5aed2af08010b2ef85}{4d9e3d8} & Taylor Fahlman & Filled general outline of produce function\\\hline
\href{https://github.com/fahlmant/cs444/commit/2244c720455b015276a94f97610bd3e710e41705}{2244c72} & Taylor Fahlman & Assigned buffer pointer correctly to the address of the item with dummy values\\\hline
\href{https://github.com/fahlmant/cs444/commit/607fa54aeb02833e23feaaebe3ca42504e86a20c}{607fa54} & Taylor Fahlman & Filled consume function\\\hline
\href{https://github.com/fahlmant/cs444/commit/f4facc9d5a16eb42168ac97078b0f487a08a407c}{f4facc9} & Taylor Fahlman & Fixed compile issues\\\hline
\href{https://github.com/fahlmant/cs444/commit/6445be50d9acb170ec3adaa80fe9418b160faf23}{6445be5} & Taylor Fahlman & Added sleep and print in consume function\\\hline
\href{https://github.com/fahlmant/cs444/commit/4b56781b7d05dd3c5653e92d8b3bfe43374a4190}{4b56781} & Taylor Fahlman & Changed the buffer to be a pointer to an array of 32 buffer items\\\hline
\href{https://github.com/fahlmant/cs444/commit/2c5734f5d2cbbf0650eefb5acbf3da30c458f406}{2c5734f} & Taylor Fahlman & Updated assignment 1\\\hline
\href{https://github.com/fahlmant/cs444/commit/5d9edbd4229b7263d39225f1ec581258af623a3d}{5d9edbd} & Taylor Fahlman & Removed unneeded semaphore .h>\\\hline
\href{https://github.com/fahlmant/cs444/commit/209f7513f40020a3ae61565dd7f115b7c72b8472}{209f751} & Taylor Fahlman & Added comments for placeholders and fixed print statement\\\hline
\href{https://github.com/fahlmant/cs444/commit/d09e23095844c4b70beb8f084938a6716eb11f24}{d09e230} & Taylor Fahlman & Changed the global buffer to a struct\\\hline
\href{https://github.com/fahlmant/cs444/commit/7ff17bdada1706eb1057ba678bde47c8f5b18774}{7ff17bd} & Taylor Fahlman & Outlined needed steps in produce\\\hline
\href{https://github.com/fahlmant/cs444/commit/c5c13a973ba6f8363baf95b9e2330ea4dca16651}{c5c13a9} & Taylor Fahlman & Changed pthread create functions to not segfault\\\hline
\href{https://github.com/fahlmant/cs444/commit/50cf97754d7e3aa081ef30f56af307d2671969ff}{50cf977} & Taylor Fahlman & Changed global buffer pointer to a variable. Changed reference to it so that it could be written to and read from\\\hline
\href{https://github.com/fahlmant/cs444/commit/050904eeb0342d93e297fec347a08544cc99ce07}{050904e} & Taylor Fahlman & Added mutex init, and locking/releasing on produce\\\hline
\href{https://github.com/fahlmant/cs444/commit/7249be01ba7edbda6723b5acbe59f46efa2886a9}{7249be0} & Taylor Fahlman & Fixed issues with printing to stdout\\\hline
\href{https://github.com/fahlmant/cs444/commit/2274e684608cf836dc48bda6d3ec5d5d9b66f5fb}{2274e68} & Taylor Fahlman & Added mutex lock/unlock to consumer\\\hline
\href{https://github.com/fahlmant/cs444/commit/29e0fd1c29a38fec3bd5a999d8bf8efa513c70fd}{29e0fd1} & Taylor & Added signal catch and reimplemented buffer as queue\\\hline
\href{https://github.com/fahlmant/cs444/commit/a4f7c992ec397d9b510126246137537b77e55930}{a4f7c99} & Taylor & typo fixes\\\hline
\href{https://github.com/fahlmant/cs444/commit/95628a0549663a27c0a3ef1f04118db28e89acda}{95628a0} & Taylor & Added signal.h so signals can work\\\hline
\href{https://github.com/fahlmant/cs444/commit/1a04be2d025bdf2f264d2fdcd2dad95ded4fb324}{1a04be2} & Taylor & Added random number generation function\\\hline
\href{https://github.com/fahlmant/cs444/commit/4d08be28f8900f197846be9ac81d7c0d4c45cc03}{4d08be2} & Taylor Fahlman & Moved function defintions to correct place\\\hline
\href{https://github.com/fahlmant/cs444/commit/a61e6320173be1ea668966396bb45bc7930d34f9}{a61e632} & Taylor Fahlman & Removed return that will never be executed\\\hline
\href{https://github.com/fahlmant/cs444/commit/12d413cdbc773bea9797fbc8ffc523df368cae5c}{12d413c} & Taylor Fahlman & Added CPUID definition and fixed compiler errors and warning about function definitions\\\hline
\href{https://github.com/fahlmant/cs444/commit/97b39ebde5ea4090c9371d7868ec9984c6230703}{97b39eb} & Taylor Fahlman & Corrected CPUID define to differ from other global variables. Added rdrand generate define\\\hline
\href{https://github.com/fahlmant/cs444/commit/056005ccb725e1ce91c211bb5cc71acdd3cfe5e4}{056005c} & Taylor Fahlman & Fixed cpuid function\\\hline
\href{https://github.com/fahlmant/cs444/commit/f927d9b39907620e959c235c9b56ab0160bbc366}{f927d9b} & Taylor Fahlman & Fixed prototype of number generator function\\\hline
\href{https://github.com/fahlmant/cs444/commit/4bcd392bf2941eb49cc82229be9495cd4c9f62ee}{4bcd392} & Taylor Fahlman & Added pthread condition variables, and fixed rdrand macro\\\hline
\href{https://github.com/fahlmant/cs444/commit/6bc6226e01a9a1302c741331cce8ab8256f399ce}{6bc6226} & Taylor Fahlman & Added blocking on full or empty buffer. Added calls to random number generation.\\\hline
\href{https://github.com/fahlmant/cs444/commit/8e614a90d9b3adfedea1214253801c36f17c6aea}{8e614a9} & Taylor Fahlman & Added mersene twister generation. Added more cleanup in the signal catch. Moved pthread create to infinite loops so that it would get more numbers from the buffer\\\hline
\href{https://github.com/fahlmant/cs444/commit/9fb3e850a20f58cd174a80bf580c150ebb8a8860}{9fb3e85} & Taylor Fahlman & Added mersene twister files\\\hline
\href{https://github.com/fahlmant/cs444/commit/007816390983ac7d562e8f9d4d31ff6abc7c40c8}{0078163} & Taylor & Removed producer sleep time because I couldn't get it to work\\\hline
\href{https://github.com/fahlmant/cs444/commit/6a7edbe421fb5e53a9a847a09117500c1ca1482e}{6a7edbe} & Taylor & Fixed blocking on full buffer\\\hline
\href{https://github.com/fahlmant/cs444/commit/c96174c136e9194fd83f4d5fee5927432468fdfd}{c96174c} & Taylor Fahlman & Fixed tabbing\\\hline
\href{https://github.com/fahlmant/cs444/commit/629da73d1a287be9db4919d9bc27c987acfb7128}{629da73} & Taylor Fahlman & Merge branch 'master' of https://github.com/fahlmant/cs444\\\hline
\href{https://github.com/fahlmant/cs444/commit/2d454270c6b79091f4c1363da7185e3ffbcb2301}{2d45427} & Taylor Fahlman & Added writeup skeleton\\\hline
\href{https://github.com/fahlmant/cs444/commit/27eb37349447092eec979e1d09531408f38ac661}{27eb373} & Taylor Fahlman & Added bash file to transform gitlog to latex. Added in makefile targets\\\hline\end{tabular}
\href{git@github.com:fahlmant/cs444/commit/1ca41740bd9248799f23d5798a4a82d99e63ba5a}{1ca4174} & Your Name & Added linux kernel\\\hline
\href{git@github.com:fahlmant/cs444/commit/15a4c12977816ab8bab1fe5f5a9b3c3774a8367a}{15a4c12} & Your Name & Added concurrency\\\hline
\href{git@github.com:fahlmant/cs444/commit/d4c0a889450f5e33eeee262f56cea203a29aed0a}{d4c0a88} & Taylor Fahlman & Makefile:     Added make file will the most common flags for style and debugging. Used a two step compile process, .o then an executable\\\hline
\href{git@github.com:fahlmant/cs444/commit/e2d0eedfaec7397cf59d7f1fd46886026f0f1727}{e2d0eed} & Taylor Fahlman & Makefile and assignment1.c update\\\hline
\href{git@github.com:fahlmant/cs444/commit/4d9e3d84c0309caf5af3ca5aed2af08010b2ef85}{4d9e3d8} & Taylor Fahlman & Filled general outline of produce function\\\hline
\href{git@github.com:fahlmant/cs444/commit/2244c720455b015276a94f97610bd3e710e41705}{2244c72} & Taylor Fahlman & Assigned buffer pointer correctly to the address of the item with dummy values\\\hline
\href{git@github.com:fahlmant/cs444/commit/607fa54aeb02833e23feaaebe3ca42504e86a20c}{607fa54} & Taylor Fahlman & Filled consume function\\\hline
\href{git@github.com:fahlmant/cs444/commit/f4facc9d5a16eb42168ac97078b0f487a08a407c}{f4facc9} & Taylor Fahlman & Fixed compile issues\\\hline
\href{git@github.com:fahlmant/cs444/commit/6445be50d9acb170ec3adaa80fe9418b160faf23}{6445be5} & Taylor Fahlman & Added sleep and print in consume function\\\hline
\href{git@github.com:fahlmant/cs444/commit/4b56781b7d05dd3c5653e92d8b3bfe43374a4190}{4b56781} & Taylor Fahlman & Changed the buffer to be a pointer to an array of 32 buffer items\\\hline
\href{git@github.com:fahlmant/cs444/commit/2c5734f5d2cbbf0650eefb5acbf3da30c458f406}{2c5734f} & Taylor Fahlman & Updated assignment 1\\\hline
\href{git@github.com:fahlmant/cs444/commit/5d9edbd4229b7263d39225f1ec581258af623a3d}{5d9edbd} & Taylor Fahlman & Removed unneeded semaphore .h>\\\hline
\href{git@github.com:fahlmant/cs444/commit/209f7513f40020a3ae61565dd7f115b7c72b8472}{209f751} & Taylor Fahlman & Added comments for placeholders and fixed print statement\\\hline
\href{git@github.com:fahlmant/cs444/commit/d09e23095844c4b70beb8f084938a6716eb11f24}{d09e230} & Taylor Fahlman & Changed the global buffer to a struct\\\hline
\href{git@github.com:fahlmant/cs444/commit/7ff17bdada1706eb1057ba678bde47c8f5b18774}{7ff17bd} & Taylor Fahlman & Outlined needed steps in produce\\\hline
\href{git@github.com:fahlmant/cs444/commit/c5c13a973ba6f8363baf95b9e2330ea4dca16651}{c5c13a9} & Taylor Fahlman & Changed pthread create functions to not segfault\\\hline
\href{git@github.com:fahlmant/cs444/commit/50cf97754d7e3aa081ef30f56af307d2671969ff}{50cf977} & Taylor Fahlman & Changed global buffer pointer to a variable. Changed reference to it so that it could be written to and read from\\\hline
\href{git@github.com:fahlmant/cs444/commit/050904eeb0342d93e297fec347a08544cc99ce07}{050904e} & Taylor Fahlman & Added mutex init, and locking/releasing on produce\\\hline
\href{git@github.com:fahlmant/cs444/commit/7249be01ba7edbda6723b5acbe59f46efa2886a9}{7249be0} & Taylor Fahlman & Fixed issues with printing to stdout\\\hline
\href{git@github.com:fahlmant/cs444/commit/2274e684608cf836dc48bda6d3ec5d5d9b66f5fb}{2274e68} & Taylor Fahlman & Added mutex lock/unlock to consumer\\\hline
\href{git@github.com:fahlmant/cs444/commit/29e0fd1c29a38fec3bd5a999d8bf8efa513c70fd}{29e0fd1} & Taylor & Added signal catch and reimplemented buffer as queue\\\hline
\href{git@github.com:fahlmant/cs444/commit/a4f7c992ec397d9b510126246137537b77e55930}{a4f7c99} & Taylor & typo fixes\\\hline
\href{git@github.com:fahlmant/cs444/commit/95628a0549663a27c0a3ef1f04118db28e89acda}{95628a0} & Taylor & Added signal.h so signals can work\\\hline
\href{git@github.com:fahlmant/cs444/commit/1a04be2d025bdf2f264d2fdcd2dad95ded4fb324}{1a04be2} & Taylor & Added random number generation function\\\hline
\href{git@github.com:fahlmant/cs444/commit/4d08be28f8900f197846be9ac81d7c0d4c45cc03}{4d08be2} & Taylor Fahlman & Moved function defintions to correct place\\\hline
\href{git@github.com:fahlmant/cs444/commit/a61e6320173be1ea668966396bb45bc7930d34f9}{a61e632} & Taylor Fahlman & Removed return that will never be executed\\\hline
\href{git@github.com:fahlmant/cs444/commit/12d413cdbc773bea9797fbc8ffc523df368cae5c}{12d413c} & Taylor Fahlman & Added CPUID definition and fixed compiler errors and warning about function definitions\\\hline
\href{git@github.com:fahlmant/cs444/commit/97b39ebde5ea4090c9371d7868ec9984c6230703}{97b39eb} & Taylor Fahlman & Corrected CPUID define to differ from other global variables. Added rdrand generate define\\\hline
\href{git@github.com:fahlmant/cs444/commit/056005ccb725e1ce91c211bb5cc71acdd3cfe5e4}{056005c} & Taylor Fahlman & Fixed cpuid function\\\hline
\href{git@github.com:fahlmant/cs444/commit/f927d9b39907620e959c235c9b56ab0160bbc366}{f927d9b} & Taylor Fahlman & Fixed prototype of number generator function\\\hline
\href{git@github.com:fahlmant/cs444/commit/4bcd392bf2941eb49cc82229be9495cd4c9f62ee}{4bcd392} & Taylor Fahlman & Added pthread condition variables, and fixed rdrand macro\\\hline
\href{git@github.com:fahlmant/cs444/commit/6bc6226e01a9a1302c741331cce8ab8256f399ce}{6bc6226} & Taylor Fahlman & Added blocking on full or empty buffer. Added calls to random number generation.\\\hline
\href{git@github.com:fahlmant/cs444/commit/8e614a90d9b3adfedea1214253801c36f17c6aea}{8e614a9} & Taylor Fahlman & Added mersene twister generation. Added more cleanup in the signal catch. Moved pthread create to infinite loops so that it would get more numbers from the buffer\\\hline
\href{git@github.com:fahlmant/cs444/commit/9fb3e850a20f58cd174a80bf580c150ebb8a8860}{9fb3e85} & Taylor Fahlman & Added mersene twister files\\\hline
\href{git@github.com:fahlmant/cs444/commit/007816390983ac7d562e8f9d4d31ff6abc7c40c8}{0078163} & Taylor & Removed producer sleep time because I couldn't get it to work\\\hline
\href{git@github.com:fahlmant/cs444/commit/6a7edbe421fb5e53a9a847a09117500c1ca1482e}{6a7edbe} & Taylor & Fixed blocking on full buffer\\\hline
\href{git@github.com:fahlmant/cs444/commit/c96174c136e9194fd83f4d5fee5927432468fdfd}{c96174c} & Taylor Fahlman & Fixed tabbing\\\hline
\href{git@github.com:fahlmant/cs444/commit/629da73d1a287be9db4919d9bc27c987acfb7128}{629da73} & Taylor Fahlman & Merge branch 'master' of https://github.com/fahlmant/cs444\\\hline
\href{git@github.com:fahlmant/cs444/commit/2d454270c6b79091f4c1363da7185e3ffbcb2301}{2d45427} & Taylor Fahlman & Added writeup skeleton\\\hline
\href{git@github.com:fahlmant/cs444/commit/27eb37349447092eec979e1d09531408f38ac661}{27eb373} & Taylor Fahlman & Added bash file to transform gitlog to latex. Added in makefile targets\\\hline\end{tabular}

\end{document}
