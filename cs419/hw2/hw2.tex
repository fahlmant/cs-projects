\documentclass{article}
\usepackage{mathtools}
\usepackage[]{mcode}

\begin{document}
\title{CS 419 Homework 2}
\author{Taylor Fahlman}
\maketitle
\begin{flushleft}
2.1\\
\end{flushleft}

The logistic growth model is defined as the following:

\begin{equation}
    X' = rX(1-\dfrac{X}{K})
\end{equation}

which is equivalent to

\begin{equation}
    \dfrac{dX}{dt} = rX(1-\dfrac{X}{K})
\end{equation}
    
We can rewrite this equation by seperating the r and X.

\begin{equation}
    r dt = \dfrac{dX}{X(1-X/K)}
\end{equation}

Before we can integrate, we need to fix the right side of the equation. We take the dX out, and use A and B as constants.

\begin{equation}
    \dfrac{A}{P} + \dfrac{B}{(1-X/K)}
\end{equation}

With a common denominator, this turns into

\begin{equation}
    A(1-X/K) + BX = A + X(B-A/K) = 1
\end{equation}

This means that

\begin{equation}
    B = \dfrac{A}{K}, and A = 1
\end{equation}

Now we can subsitute A and B back in:

\begin{equation}
    \dfrac{1}{X} + \dfrac{1/K}{(1-X/K)}
\end{equation}

This simplifies back to 

\begin{equation}
    r dt = \dfrac{dX}{X} + \dfrac{dX/K}{(1-X/K)}
\end{equation}

Now we interegrate both sides:

\begin{equation}
    \int r dt = \int \dfrac{dX}{X} + \int \dfrac{dX/K}{(1-X/K)}
\end{equation}

\begin{equation}
    rt + c = ln(X) + \int \dfrac{dX/K}{(1-X/K)}
\end{equation}

If we use u = 1 - X/K and du = -1/X dX for integration with substitution, it turns out to be

\begin{equation}
    \int \dfrac{-du}{u} = -ln(u) = -ln(1-X/K)    
\end{equation}

Which can be rewritten as

\begin{equation}
    rt + c = ln(X) - ln(1-X/K) = ln(\dfrac{X}{(1-X/K)})
\end{equation}

Now to eliminate the ln, we can take both sides to the e. We use C to represent the constant.

\begin{equation}
    Ce^{rt} = \dfrac{X}{1-X/K}
\end{equation}

Now we can solve for X

\begin{equation}
    (1-\dfrac{X}{K})Ce^{rt} = X
\end{equation}
\begin{equation}
    Ce^{rt} = X(1+\dfrac{Ce^{rt}}{K})
\end{equation}
\begin{equation}
    X = \dfrac{Ce^{rt}}{1+\dfrac{Ce^{rt}}{K}} 
\end{equation}

2.3

\begin{lstlisting}
function xchange = logistic(t, tau)

    r = 1;
    k = 1000;
    x = calcx(r, t, k);
    xtau = calcx(r, (t-tau), k);
    h = 1;
    
    xchange = r*x*(1-x/k)-(h*xtau);

function x=calcx(r, t, x)

    x = (exp(r*t)/(1+(exp(r*t)/k)));
\end{lstlisting}

If we run this script with the following code we can see our results

\begin{lstlisting}
tspan = [0 20]
x0 = 1
[t,x] = ode45('logistic', tspan, tau)
plot(t, x, xlabel('Time'), ylabel('Population Density'))
\end{lstlisting}

After running this model with various values of tau, we see as the value of tau gets larger, the carrying capacity of the population becomes lower and lower.  
\end{document}
