\documentclass{article}
\usepackage{mathtools}

\begin{document}
\title{CS 419 Homework 2}
\author{Taylor Fahlman}
\maketitle
\begin{flushleft}
2.1\\
\end{flushleft}

The logistic growth model is defined as the following:

\begin{equation}
    X' = rX(1-\dfrac{X}{K})
\end{equation}

which is equivalent to

\begin{equation}
    \dfrac{dX}{dt} = rX(1-\dfrac{X}{K})
\end{equation}
    
We can rewrite this equation by seperating the r and X.

\begin{equation}
    r dt = \dfrac{dX}{X(1-X/K)}
\end{equation}

Before we can integrate, we need to fix the right side of the equation. We take the dX out, and use A and B as constants.

\begin{equation}
    \dfrac{A}{P} + \dfrac{B}{(1-X/K)}
\end{equation}

With a common denominator, this turns into

\begin{equation}
    A(1-X/K) + BP = A + P(B-A/K) = 1
\end{equation}

This means that

\begin{equation}
    B = \dfrac{A}{K}, and A = 1
\end{equation}

Now we can subsitute A and B back in:

\begin{equation}
    \dfrac{1}{X} + \dfrac{1/K}{(1-X/K)}
\end{equation}

This simplifies back to 

\begin{equation}
    r dt = \dfrac{dX}{X} + \dfrac{dX/K}{(1-X/K)}
\end{equation}

Now we interegrate both sides:

\begin{equation}
    \int r dt = \int \dfrac{dX}{X} + \int \dfrac{dX/K}{(1-X/K)}
\end{equation}

\begin{equation}
    rt + c = ln(X) + \int \dfrac{dX/K}{(1-X/K)}
\end{equation}

If we use u = 1 - X/K and du = -1/X dX for integration with substitution, it turns out to be

\begin{equation}
    \int \dfrac{-du}{u} = -ln(u) = -ln(1-X/K)    
\end{equation}
    
%    ln(\dfrac{X}{1-X/K})

\end{document}
