\documentclass{article}
\usepackage{mathtools}

\begin{document}
\title{Modeling Acoustic Propagation in Fluids\\
CS 419}
\author{Taylor Fahlman}
\maketitle

\section{Introduction}

Sound surrounds us every day. It is important to our everyday lives; we use hearing and sound for survival,
productivity, and enjoyment. Sound is one of the most important aspects of living. Because of this, the 
understanding of sound and modeling its various properties today is important. Researchers use various
models of sound to understand more finely its characteristics. How sound propagates in fluids, specifically
water and air in most cases, can provide insight into a number of different systems. For example, modeling
acousitcs in a concert hall or similar musical venue can show engineers how best to build the room to
optomize the musical or otherwise acoustic result of the performance. Another very important example
in medicine is ultrasounds. Modeling how sound can propagate through fluids (and other mediums) enable
doctors and software engineers to accurately represent and interpret the results of an ultrasound
procedure. A third example is being able to locate objects underwater. The military uses models
of sound propagation in water to locate submarines in the ocean while on patrol. Similarly,
researchers use models of sound propagation to locate and identify whale pods. Each pod uses
a unique frequencey and patterns to communicate, a kind of pseudo-language. Understanding how
various frequencies move in different ways through water help researchers track migration patters,
proximity of pods, and various other information vital to their research. These are only a few
examples of sound modeling. There are many various fields these methods can be applied to. These
example show the importance of sound modeling, as well as the wide range of applications it can have.

\section{Previous Work}
Due to the wide array of applications, there are many methods to modeling sound, and each has certain
aspects it focuses on. For example, some models focus on the density of the fluid, while others factor
in temperature. There are also different methods for modeling one-dimensional, two-dimensional and
three-dimensional propagations. All of these various models make it difficult to find exact one needed
for a given situation. The variability in application also means a variability in complexity. This means
that overall, it is hard to find and compute the precise solution one wants. 


\section{Contribution}

\section{Prediction}

\section{Sources}

\end{document}
