\documentclass{article}

\begin{document}
\title{CS 419 Homework 1}
\author{Taylor Fahlman}
\maketitle
\begin{flushleft}
1.4\\
\end{flushleft}
The trapezoidal method is defined as:


\begin{equation}
    w_n = w_{n+1} + (h/2)(f(t_n, w_n) + f(t_{n+1}, w_{n+1}))
\end{equation}


The stability of this rule with consideration of the amplification factor can be defined as:

\begin{equation}
    Q(h\lambda) = (1+(h/2)\lambda)/(1-(h/2)\lambda)
\end{equation}

Where stability is defined as:

\begin{equation}
    lim_{h->0}|Q(h\lambda)|< = 1
\end{equation}

And we get our Q function from
\begin{equation}
    w_{n+1} = Q(h\lambda)w_n = (Q(h\lambda))^{n+1}y_0
\end{equation}

This means that for the trapezoidal rule to be stable, h must always be greater than 0. If this is not the case,
then the defenition of unstable would be true, which is:

\begin{equation}
    lim_{h->0}|Q(h\lambda)|> 1
\end{equation}

\begin{flushleft}
1.9
\end{flushleft}

Convergence in the case of integration approximation can be defined as: \\

stability + consistency = convergence\\

So we need to estimate stability and consistency to prove convergence. We have
already solved for stability above. For consistency, we'll need the local truncation error.
The gerenal form of local truncation error is:

\begin{equation}
    \tau_{n+1}(h) = (y_{n+1} - z_{n+1})/h
\end{equation}

And consistency is

\begin{equation}
    \lim_{h->0}\tau_{n+1} = 0
\end{equation}

So if we take the expanded taylor series of the trapezoidal rule:

\begin{equation}
    (1+(h/2)\lambda)/(1-(h/2)\lambda) = 1 + h\lambda + (1/2)(h\lambda)^2 + (1/4)(h\lambda)^3 + O(h^4)
\end{equation}
And plug that into the truncation error
\begin{equation}
    \tau_{n+1}(h) = ((e^{h\lambda} - Q(h\lambda))y_n)/h = (1/h)(-(1/6)(h\lambda)^3 + O(h^4))y_n    
\end{equation}
\begin{equation}
    (-1/6)h^2\tau^3 + O(h^3))y_n = \tau_{n+1}(h) = O(h^2)
\end{equation}

The local truncation error is on the order of two. So the trapezoidal rule is a second order method.\\

If we take the same equation for the Forward Euler method, where 
%\begin{equation}

%\end{equation}

%\begin{equation}

%\end{equation}
%\begin{equation}

%\end{equation}


\end{document}
