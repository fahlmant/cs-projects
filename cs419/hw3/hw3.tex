\documentclass{article}
\usepackage{mathtools}
\usepackage[]{mcode}

\begin{document}
\title{CS 419 Homework 3}
\author{Taylor Fahlman}
\maketitle
\begin{flushleft}
3.1\\
\end{flushleft}

\begin{lstlisting}
n = 10;
h = 100;
k = .1;
m1=ones(n-2,1);
A = spdiags([m1 -2*m1 m1], [-1 0 1], n-2, n-2);
b=zeros(n-2,1);
b(1,1) = 0;
b(n-2,1)=10;
tau = 1/abs(eig(A));
for i=1:n-2
    xj = (h/2) + ((i-1) *h);
    u0(i) = 10*xj+5*sin(2*pi*xj)*sin(2*pi*xj);
end

rhs = @(t,u) k*(1/h^2)*A*u+k*(1/h^2)*b;

tic
[Ts, Xr] = ode45(rhs, [0:0.01:1], u0);
toc

figure(1), semilogy(1, tau), hold on;
xlabel('t');
ylabel('tau');
title("Tau values");

\end{lstlisting}
Using various values of n, we see that the higher that n gets, the closer and closer our
time constant values approach .25 in the upper left value of tau. This is because our coeffecient matrix helps approximate
the temperature of each part more accurately if there are more parts to measure. 
The system seems to be decently stiff, with only the lowest values of n varying much, but
as n approaches 100, the system seems to converge relatively quickly toward one solution.

\end{document}
